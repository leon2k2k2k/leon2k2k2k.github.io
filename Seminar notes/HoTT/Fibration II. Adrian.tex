\documentclass[../main.tex]{subfiles}

\title{Fibration part II}
\author{Adrian Clough}
\date{March 2nd, 2020}

\begin{document}
\maketitle

\noident Today is going to be a Q&A session over fibrations, led by Adrian Clough. 
\tableofcontents

\section{Last time}

Recall that $p: E \rightarrow B$ is a fibration if the fiber vary nicely when you move around on the base. This is a general context in mathematics. In a topological setting, one example of varying nicely is called a vector bundle:

Q: what is a vector bundle.

A: A vector bundle over $B$ is something that locally looks like $\mathbb{R}^n \times B$ over $B$. 

One example would be the cylinder and Mobius strip vector bundle over $S^1$. Note that we have the infinite Grassmannian $Gr_{n,\infty} = \{V \subset \mathbb{R}^\infty|dim V = n\}$, then it turns out that a vector bundle over $B$ is the "same thing" as a map from $B$ to $Gr_{n,\infty}$. 

This same game is played when we think about general fibration. 

Now we are going to tell the "Kan" version:

\begin{definition}
Let $E$ and $B$ be groupoids and a map $p: E \rightarrow B$, then a map $f: e \rightarrow e'$ in $E$ is called $p$-coCartesian if: 

we know that we have $pf: pe \rightarrow pe'$ downstairs, and $fe \rightarrow e'$ lives over that. For all maps $g: e \rightarrow e''$ upstairs, which projects down to $pg: e \rightarow e''$, and all map $\hat{h}: pe' \rightarrow pe''$ downstairs such that $pg = h \circ pf$, then exists unique map $h: e' \rightarow e''$ that maps down to it. 
\missingfigure{I need to draw the classic p-cartesian morhphism diagram}

A map $p: E \rightarrow B$ is called cocartesian if for all every $e \in E$, every $\hat{f} : pe \rightarrow b$, then exists p-coCartesian $e \xrightarrow{f} e' : pf = \hat{f}$.
\end{definition}





\end{document}