%! Author = liule
%! Date = 8/9/2020

\documentclass[12pt]{article}%%<-this describes the class of the document. 12pt refers to the size, there are 3 options, 10,11,12. There are other parameters for article for you to select, but that's not very important.

%-< This is the list of all the packages I am using:
\usepackage[utf8]{inputenc}%%<- This introduces a bunch of symbols.
\usepackage{comment}%<- this allows you to do multi-line of comments
\usepackage{ragged2e}%<- allows to use justifying, which is ragged left and right.
\usepackage{indentfirst}%<- This requires the first paragraph to be indented.
\usepackage{amsthm}%<- Introduces unnumbered theorem environment for remark, comments, etc.
\usepackage{enumitem}
\usepackage{graphicx}
\usepackage{amsmath}%<- gives you flexibility in math display mode
\usepackage{amssymb}
\usepackage{amsthm}
\usepackage{parskip}
\usepackage{xcolor}
\usepackage{tikz}
\usepackage{tikz-cd}
\usepackage{mathtools}
\usepackage{mathalfa}
\usepackage[colorinlistoftodos]{todonotes}  %%<- this creates to-do notes.

%<- New for all the parameters and basic styling:
\setlength{\parindent}{1em} %<- this sets the identation of every paragraph
\renewcommand\qedsymbol{$\blacksquare$}%<- This sets the QED at the end of the proofs to be blacksquares
\newtheorem{theorem}{Theorem}[section] %<- first {??} is the data type, second {} is what will be shown.[] tells you when does it starts recounting.
\newtheorem{corollary}{Corollary}[section]
\newtheorem{lemma}{Lemma}[section]
\theoremstyle{definition}
\newtheorem{definition}{Definition}[section]
\theoremstyle{remark}
\newtheorem*{remark}{Remark} %<- * means that there is no numbering.
\theoremstyle{definition}
\newtheorem*{example}{Example}
\theoremstyle{definition}
\newtheorem*{exercise}{Exercise}

% There is where I set all the macros:
\newcommand{\zto}{\lim_{z\mapsto 0}}
\newcommand{\ozz}{\mathcal{O}(z,\bar{z})}
\newcommand{\curlyo}{\mathcal{O}}
\newcommand{\ztoinf}{\lim_{z \mapsto \infty}}
\newcommand{\stateo}{|\mathcal{O}>}



\title{String theory ??} %<- The title
\author{Jacque Distler} %<- The author
\date{2/24/2020} %<- The date.


\begin{document}
    \maketitle %%<- this tells the document to make the title right here. You can also create a title page.

    \noindent % This is the introduction paragraph of the talk.

    \tableofcontents %<- this allows you to make the table of contents.

    \section{Last time}

    Last time we had that $T_{zz} = \sum_n ^{-n-2} L_n$, with $z = e^w$, then $T_{ww} = \sum_n (L_n - \mu_0 \delta_{n,0}e^{-nw}$. $z$ is the cylinder quantization and $w$ is the standard Minkowski plane. From this we calculated that $\mu_0 = c/24$. We also have OPE
    $$ T(z') T(z) = \frac{c/2}{(z'-z)^4} + \frac{2T(z)}{(z'-z)^2} + \frac{\partial_z T}{z' -z} + ...$$, where $...$ is something regular, thus doesn't contribute to line integrals.
    Note that the vacuum is $sl_2$ invariant.

    \section{Extending to the Riemann Sphere}

    \subsection{Extends to the origin}

    Extending to orogin z = 0: $lim_{z \mapsto 0} \mathcal{O}(z,\bar{z}) |0>$ is finite. In particular, for $T(z) = \sum_n z^{-n-2} L_n$, $lim_{z \mapsto 0} T(z) |0>$ finite is equivalent to $L_n |0> = 0$ for $n \geq -1$. Note that this automatically implied that the vaccum is $sl_2$ invariant.
    This was automatic for free scalar, with $a^\dagger n = a_{-n}, n \geq 0$, and $a_n |0> = 0, n \geq 0$,
    and our normal ordering convention $$L_n = 1/2 \sum_{m = -\infty}^\infty :a_{n-m}a_m:$$ \todo {normal ordering symbol I don't know}. From this, we see that $lim_{z\mapsto 0} T(z) = L_{-2} |0> = 1/2 a^2_{-1}|0>$, more generally, $lim_{z\mapsto 0} \mathcal{O}(z,\bar{Z}) |0> \coloneqq |mathcal{O}>$, this is the operator-state correspondence.

    Now we can say anything about operators to thing we can say about the states:

    If $\mathcal{O}$ is primary, we have $[L_n, \mathcal{O}z,\bar{z}] = (n+1)z^n h \mathcal{O}(z,\bar{z}) + z^{n+1} \partial_z \mathcal{O}(z, \bar{Z})$, now we take the limit as $z \mapsto 0$: for $n \geq 1$, $lim{z\mapsto 0} [L_n, \mathcal{O}(z,\bar{z}) |0> = L_n |\mathcal{O}> = 0$ the last part is by the calculation above about the commutator. $\zto [L_0, \ozz]|0> = L_0 |\mathcal{O}> = h|\mathcal{O}>$, and $\zto [L_{-1}, \ozz] |0> = |\partial_z \curlyo>$. States that satisfies this are called the primary states.

    Similiarly, we have descendent states, where $L_{-1}$ doesn't vanish as in the primary states, one example of this would be $\partial_z \curlyo$, which by Jacobi identity doesn't vanish. They all look like this, this is why they are called descendent states.

    \subsection{extending z = $\infty$}

    Let $y = -1/z$, in changing of coordinates, we see that $T_{yy}= \frac{dz}{dy}^2 T_{zz}(z) + c/12 \{z,y\} = z^4T_{zz}(z)$. $\{\}$ is the Scharwzian derivative \todo{I don't know how to spell it}.

    That tells me how the stress-energy tensor transform:

    So given a correlation function$<T(z) \matcal{O}_1(z_1, \bar{z}_1)...\mathcal{O}_n(z_n, \bar{z}_n)> ~ z^{-4}$ as $z \mapsto \infty$, the $z^{-4}$ is there to ensure that it cancels the $z^4$ from the stress energy tensor.

    Thus we define $<\mathcal{O}| \coloneqq <0| lim_{z\mapsto \infty} z^{2L_0}\bar{z}^{2\tilde{L}_0}$, $<0|L_n = 0 for n \leq 1$.

    Note that for a primary operator $\mathcal{O}(y,\bar{y}) = \frac{dz}{dy}^h (d\bar{z})(d\bar{y})\ozz$.

    Sow we learn that

    \begin{enumerate}
        \item states are organized into representations of Virasoro. We have the primary states $|mathcal{O}>$ with $L_0 \stateo = h \stateo$, $L_n \stateo = 0 , n \geq 1$. And it has descendents $\ldots L_{-3}^{n_3} L_{-2}^{n_2}L_{-1}^{n_1} \stateo$, then $sym_{k=1}^{\infty}kn_k = N$, $N$ is called the level, and $L_0$ has this a eigenstate with eigenvalue $=h+N$.
        \item  Unitarity: fro any state (primary or descendent), we have $<\psi| \psi> \geq 0$, if it is =0, then it is called a null state, corresponding operator vanishes by the equation of motion.
        Since $L^\dagger_0 = L_0 \Rightarrow h$ is real.
        $|L{-1}\stato|^2 = <\stateo| L_1 L_{-1} |\stateo> =  2h<\stateo|\stateo>$, thus $h \geq 0$, $h = 0$ iff $l_{-1}\state0 = |\partial_z \mathcal{O}> = 0$, which is equivalent to $\mathcal{O} = \mathcal{\bar{z}}.$ One example of this is $\bar{T}_{\bar{z} \bar{z}}$, which has $[L_{-1},  \bar{T}_{\bar{z} \bar{z}}(\bar{z})] = 0$. So we see an example of null state: if $\mathcal{O}$ is a primary operator with $h = 0$, then its descendent $\partial_z \mathcal{O}$ has the corresponding state a null state.


    \end{enumerate}

    What about null state at level two: there are more conditions (it will involve both $c$ and $h$):
    Let's start with a primary state, study the matrix of inner products of descendants at level two:

    \todo[inline]{I don't know how to write matrices}

    $A = (<\mathcal{O}| L_2 L_{-2} | \mathcal{O}> <\mathcal{O}|L_2 L^2_{-1}| \mathcal{O}>...$. The calculation goes that

    \todo[inline]{I also don't know how to quickly to bra-ket symbols, as well as operators of a Hamiltonian}

    $[L_2, L_{-2}] = 4 L_0 + c/2$, and $[L^2_1, L^2_{-1}] = 4L_0(2 L_0 +1)$.

    We will get more and more conditions at each level. The goal is write out a condition for the zero being the determinant of the $n$-th descendent of a primary.

    Why do we care about the determinant being positive? Because if this inner product is positive definite, then the determinant better be positive.

    \listoftodos
\end{document}