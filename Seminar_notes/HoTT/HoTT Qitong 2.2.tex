\documentclass[12pt]{article}%%<-this describes the class of the document. 12pt refers to the size, there are 3 options, 10,11,12. There are other parameters for article for you to select, but that's not very important.

%-< This is the list of all the packages I am using:
\usepackage[utf8]{inputenc}%%<- This introduces a bunch of symbols.
\usepackage{comment}%<- this allows you to do multi-line of comments
\usepackage{ragged2e}%<- allows to use justifying, which is ragged left and right.
\usepackage{indentfirst}%<- This requires the first paragraph to be indented.
\usepackage{amsthm}%<- Introduces unnumbered theorem environment for remark, comments, etc.
\usepackage{enumitem}
\usepackage{graphicx}
\usepackage{amsmath}%<- gives you flexibility in math display mode
\usepackage{amssymb}
\usepackage{amsthm}
\usepackage{parskip}
\usepackage{xcolor}
\usepackage{tikz}
\usepackage{tikz-cd}
\usepackage{mathtools}
\usepackage{mathalfa}
\usepackage[colorinlistoftodos]{todonotes}  %%<- this creates to-do notes.

%<- New for all the parameters and basic styling:
\setlength{\parindent}{1em} %<- this sets the identation of every paragraph
\renewcommand\qedsymbol{$\blacksquare$}%<- This sets the QED at the end of the proofs to be blacksquares
\newtheorem{theorem}{Theorem}[section] %<- first {??} is the data type, second {} is what will be shown.[] tells you when does it starts recounting.
\newtheorem{corollary}{Corollary}[section]
\newtheorem{lemma}{Lemma}[section]
\theoremstyle{definition}
\newtheorem{definition}{Definition}[section]
\theoremstyle{remark}
\newtheorem*{remark}{Remark} %<- * means that there is no numbering.
\theoremstyle{definition}
\newtheorem*{example}{Example}
\theoremstyle{definition}
\newtheorem*{exercise}{Exercise}

% There is where I set all the macros:




\title{Transportation and Path Lifting} %<- The title
\author{QiTong} %<- The author
\date{2/24/2020} %<- The date. 


\begin{document}
\maketitle %%<- this tells the document to make the title right here. You can also create a title page.

\noindent % This is the introduction paragraph of the talk.

\tableofcontents %<- this allows you to make the table of contents.

\section{Path lifting} 

From section 2.2 of the book, we have for any $f: A \rightarrow B$ for any $x,y:A$, we have $ap_f: (x =_A y) \rightarrow f(x) =_B f(y)$. But today we want to extend this to a dependent type $f: \Pi_{(x :A)} B(x)$, and $p: x = y$, we want to say taht $f(x) = f(y)$, but the problem is fundamentally they live in different types, those saying that doesn't even make sense.

Ths idea is to use the indescernibility of the identicals:

For every $C : A \rightarrow U$, and $f : \Pi_{(x,t :A)} \Pi_{(p : x =_A y)} C(x) \rightarrow C(y)$, such taht $f(x,x,refl_x) : \equiv id_{C(x)}$. Note that given $p : x = y$, then we have $p_* : P(x) \rightarrow P(y)$, we denote this as $transport^P(p,-): P(x) \rightarrow P(y)$. We call this transport.

There is also a transportation lemma that is called path lifting.

\begin{remark} 
Q: What is a fibration, what is a fiber? 
A: A good example of fibration is the cylinder project down to the circle.

\todo[inline]{maybe I should write something that explains this a little bit}

\end{remark}

Given $P: A \rightarrow \matcal{U}$, then we can construct $pr:\Sigma_{x:A} P(x) \rightarrow A$, $\Sigma_{x:A}P(x)$ is called the total space, and $A$ is the base space. 

\begin{remark}
This is called the Grothendieck construction, that relates fibrations and maps into the moduli space of all types. This is also called the straightening/unstraightening principle.
\end{remark}

\missingfigure{There is a path lifting figure in the text book}

Given $p: x = y$, $u : P(x)$, and $P_*(u)$ is the endpoint. Now since now we have parallel transport, we can transport one to another and then we can ask are they equal or not.

Given $P: A \rightarrow \mathcal{U}$, and assuming that we have $u : P(a)$ for some $x : A$, and given a path $p : x = y$, we define $lift(y,p): (x,u) = (y, p_*(u)$ in $\sigma_{(x: A)} P(x)$, such that $pr_1(lift(u,p)) = p$.

In classical homotopy theory, there exists lifting/ path and fibration. In this version, there is a canonical one. This also shows that type family are fibrations.

Now we define $f': A \rightarrow \Sigma_{x:A} P(x)$, $f'(x) :\equiv (x, f(x))$, $f'(p): f'(x) = f'(y)$, such that $pr_1 \circ f \equiv id_A$, and $pr_ (f'(p)) = p$.

The path $lift(u,p)$ from $(x,u)$ to $(y, p_*(u))$ lies over $p$, thus any path from $u:P(x)$ to $v : P(u)$ over P factors through $lift(u,p)$ by a path from $p_*(u)$ to $v$ lying over the fiber $p(y)$.



\listoftodos




\end{document}