\documentclass[../main.tex]{subfiles}

\title{Counterexamples to HKR in Characteristic p}
\author{Rok Gregoric}
\date{2/28/2020}

\begin{document}

\maketitle


\noindent We will continue from last time, looking at how would HKR work in characteristic $p \neq 0$.



\tableofcontents

\section{Last time}

Let $k$ be a field, and $A$ is a $k$-algebra. We introduce the Hochschild complex: $HH(A/k) \coloneqq A \otimes_{A \otimes A} A$ (note all tensor products now are derived), and the $i$-th homology of $HH(A/k)$ is called the $i$-th homology.

We can also generalize this to $X$ a smooth scheme, and we have the weak HKR isomorphism theorem:

\begin{theorem}
If $X/k$ is a smooth scheme, then $H^{-i}(HH(X/k)) \cong \Omega^i_{X/k}$.


\end{theorem}

Question: What does this tell us about $HH_i(X/k)$?

In a general setting: if $\mathcal{F}$ be a chain complex of quasi-coherent sheaves on $X$. What is the relationship between $H^s(X;H^t(\mathcal{F}))$ relate to $H^?(X, \mathcal{F}) \coloneqq H^?(R\Gamma(X;\mathcal{F}))$.

A: They are related by a spectral sequence.

We might guess that: $HH(X/k) = \sum_{s-t=n} H^t(X; \Omega^s_X)$.

This is called the Hodge decomposition. In characterstic 0, we always gets this result, which is the strong HKR that we discuss last time:

\begin{theorem}
$HH(X/k) \cong \bigoplus_{i \geq 0} \Omega^i_X[i]$ if $X$ is smooth, if not, we replace $\Omega$ with the cotangent complex: $HH_n(X/k) \overline{=} \bigoplus_{s-t =n}H^2(X;\wedge^tL_{X/k})$.
\end{theorem}

This is related to the Hodge decomposition of Kahler manifolds. But in there there is a $S^1$ action, periodic cyclic homology, which we are not discussing today.


\section{characteristic p}

Q: when char $k = p > 0$ a perfect field, does the Hodge decomposition of $HH$ still exist?

A: Yes, if $X^d$ smoioth projective over $k$, and $d \leq p$. No, in general. The no situation is what we will discuss today.

\begin{theorem} [Antieau-Bhatt-Mathew,'19]
There exists a smooth projective $2p-dim$ scheme $X/k$ such that Hodge decomposition for HH does not hold for $X$.


\end{theorem}

The proof comes in two steps:
1. Find a classifying stack counterexample.
2. Approximate by smooth scheme.

Let's do step one:

We are going to look at classfiying stack $BG$, where $G$ is finite. In our case, $G = \mu_p$ the group scheme of roots of unity. $\mu(R) = \{x \in R| x^p = 1\}$. 

Rough sketch of 2:
$V$ a f.d. $G$ representation, consider $\mathbb{P}(V)$. By a Bertini-type theorem: there exists $X \subset \mathbb{P}(V)$ smooth complete intersection, such that $G$ act on $X$ freely:

We have $X/G \cong [X/G] \hookrightarrow [\mathbb{P}(V)/G] \rightarrow BG$. We have $X/G$ a smooth projective scheme, the inclusion is a smooth locally complete intersection, and the last map a projective bundle. Because it is a projective bundle, the pullback on cohomology is injective. Because Hodge decomposition fails for $BG$, and the Hodge decomposition fails for $[\mathbb{P}(V)/G]$. Because of Lefschetz principle, $X/G$ would have the same lower cohomology as $[\mathbb{P}(V)/G]$, thus we see that Hodge decomposition fails for $X/G$.

\begin{remark}[Sam]
The agreement of the cohomology in low degrees is based on Lefschetz principle. The idea of doing this kind of manuveur from $BG$ is from Serre. The idea is that $BG$ for $G$ a finite group has no rational cohomology, only torsion. Thus pulling back from $BG$ will give you a lot of torsion cohomology classes.
\end{remark}

Now we are going to discuss step 1:

\begin{remark}[Sam]
This is from talking with Bhargav. We want to compare $HH_*(B\mu_p)$ and $H^*(B\mu_p, \wedge^*L_{B\mu_p})$. We want to show that RHS is not the direct sum of things on the left hand side. We are going to show that the LHS essentially vanishes. 

Idea for LHS: $Qcoh(BG) = Rep G$. If $G = \mu_p$, what are the representations? We have the trivial rep, and the canonical one ($\mu_p \in \mathbb{G}_m$). We will call this $k(1)$, then if we denote $n$ tensor of them to be $k(n)$, then we see that $k(p) = triv$. 

Claim: Rep($\mu_p$) = $\mathbb{Z}/p$-graded vector spaces.
Rep($\mu_p$) = $\mathcal{O}(\mu_p)$-comod = $\mathcal{O}(\mu_p)^{*}$-mod, and $\mathcal{O}(\mu_p) \cong \Pi^{p-1}_{i=0} k$. \todo{in display mode}

We see that $HH$ depends only on the category of quasicoherent sheaves. $HH(B\mu_p) = \Pi^{p-1}_{i=0} k$.

\end{remark}

Now we want to show that $H^*(B\mu_p, \wedge^*L_{B\mu_p})$ is huge.

We will do this calculation for arbitrary $G$.

Given $*\rightarrow{e} G$, then we define $coLin(G) \coloneqq e^*L_{G/k}$, which has a $G$ action, which is the coadjoint representation. If $G$ is smooth, then $coLie(G) = \mathfrak{g}^*$. $G = \mu_p$ is not smooth in characterstic $p$. We have a lemma:

\begin{lemma}
$R\Gamma(BG; \wedge^iL_{BG}) \cong R\Gamma(G;coLie(G)[-i]) = Sym^i(coLie(G))^G[-1]$
\end{lemma}

\begin{proof}

\missingfigure{I need to draw a pullback diagram}
\todo[inline]{I need to draw out the BG pullback diagram}

$coLie(G) = e^*L_{G/k} = e^* p^* L_{*/BG} = L_{*/BG}$, due to left exactness of $L$, we see that $* \rightarrow{BG} \rightarrow *$ gives us a fiber sequence: $\pi^*L_{BG/k} \rightarrow L_{*/*} = 0 \rightarrow L_{*/BG} \Rightarrow L_{*/BG} \cong \pi^*L_{BG/k}[1]$. 

We know that $\pi^*$, when interpreted as from $Rep(G) \rightarrow Vect_k$, it is the forgetful map. 

Thus we see that $Sym^i(coLie(G) \cong Sym^i(L_{BG/k}[1]) \cong \wedge^iL_{BG/k}[i]$. The last thing is the thing that we are trying to calculate. For $G = \mu_p$, we want to calculate the invariance of this representation.

For $G = \mu_p = spec k[t]/(t^p -1)$, then we have a two term resolution $k[t] \rightarrow{t \mapsto t^p -1} k[t]$. Pass to $L$: $k[t]dt \rightarrow k[t]dt$ sends $dt \mapsto d(t^p-1) = pt^{p-1}dt = 0$. 

We see that $e^*L_{\mu_p/k} \cong k \oplus k[1]$. 

Now we see that $H^*(BG; \wedge^*BG) = \wedge^*_k(d) \otimes Sym_k(c)$. 

Note that in our context, taking the invariance is exactly taking the degree 0 case. In our case, this representation is trivial because $G$ is abelian thus the co-adjoint representation is concentrated in the 0 case.



\end{proof}

\listoftodos

\end{document}